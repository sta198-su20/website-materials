% Options for packages loaded elsewhere
\PassOptionsToPackage{unicode}{hyperref}
\PassOptionsToPackage{hyphens}{url}
%
\documentclass[
]{article}
\usepackage{lmodern}
\usepackage{amssymb,amsmath}
\usepackage{ifxetex,ifluatex}
\ifnum 0\ifxetex 1\fi\ifluatex 1\fi=0 % if pdftex
  \usepackage[T1]{fontenc}
  \usepackage[utf8]{inputenc}
  \usepackage{textcomp} % provide euro and other symbols
\else % if luatex or xetex
  \usepackage{unicode-math}
  \defaultfontfeatures{Scale=MatchLowercase}
  \defaultfontfeatures[\rmfamily]{Ligatures=TeX,Scale=1}
\fi
% Use upquote if available, for straight quotes in verbatim environments
\IfFileExists{upquote.sty}{\usepackage{upquote}}{}
\IfFileExists{microtype.sty}{% use microtype if available
  \usepackage[]{microtype}
  \UseMicrotypeSet[protrusion]{basicmath} % disable protrusion for tt fonts
}{}
\makeatletter
\@ifundefined{KOMAClassName}{% if non-KOMA class
  \IfFileExists{parskip.sty}{%
    \usepackage{parskip}
  }{% else
    \setlength{\parindent}{0pt}
    \setlength{\parskip}{6pt plus 2pt minus 1pt}}
}{% if KOMA class
  \KOMAoptions{parskip=half}}
\makeatother
\usepackage{xcolor}
\IfFileExists{xurl.sty}{\usepackage{xurl}}{} % add URL line breaks if available
\IfFileExists{bookmark.sty}{\usepackage{bookmark}}{\usepackage{hyperref}}
\hypersetup{
  pdftitle={STA 102: Intro to Biostatistics},
  pdfauthor={Spring 2020},
  hidelinks,
  pdfcreator={LaTeX via pandoc}}
\urlstyle{same} % disable monospaced font for URLs
\usepackage[margin=1in]{geometry}
\usepackage{graphicx,grffile}
\makeatletter
\def\maxwidth{\ifdim\Gin@nat@width>\linewidth\linewidth\else\Gin@nat@width\fi}
\def\maxheight{\ifdim\Gin@nat@height>\textheight\textheight\else\Gin@nat@height\fi}
\makeatother
% Scale images if necessary, so that they will not overflow the page
% margins by default, and it is still possible to overwrite the defaults
% using explicit options in \includegraphics[width, height, ...]{}
\setkeys{Gin}{width=\maxwidth,height=\maxheight,keepaspectratio}
% Set default figure placement to htbp
\makeatletter
\def\fps@figure{htbp}
\makeatother
\setlength{\emergencystretch}{3em} % prevent overfull lines
\providecommand{\tightlist}{%
  \setlength{\itemsep}{0pt}\setlength{\parskip}{0pt}}
\setcounter{secnumdepth}{-\maxdimen} % remove section numbering
% https://github.com/rstudio/rmarkdown/issues/337
\let\rmarkdownfootnote\footnote%
\def\footnote{\protect\rmarkdownfootnote}

% https://github.com/rstudio/rmarkdown/pull/252
\usepackage{titling}
\setlength{\droptitle}{-2em}

\pretitle{\vspace{\droptitle}\centering\huge}
\posttitle{\par}

\preauthor{\centering\large\emph}
\postauthor{\par}

\predate{\centering\large\emph}
\postdate{\par}

\title{STA 102: Intro to Biostatistics}
\author{Spring 2020}
\date{}

\begin{document}
\maketitle

STA 102 is an introductory course in statistics and data science
motivated by timely applications from the health sciences, biomedical
research, and public health. Students will understand common statistical
methods and their suitability in answering specific research questions
of interest, conduct rigorous, reproducible analysis using R, interpret
results in context and translating them to language accessible to allied
health science researchers, and critique statistical usage in the field
in order to evaluate data-based claims and decisions.

\hypertarget{activities-assessments}{%
\subsubsection{Activities \& Assessments}\label{activities-assessments}}

Activities and assessments focus on understanding methods and
interpreting results, as well as hands-on analyses of real-world data
using R. All homeworks and labs must be made via electronic submission
to the class Sakai page using the R Markdown templates provided, with
unlimited re-submissions allowed until noon on the due date (only the
most recent version will be graded).

\hypertarget{homework-20}{%
\paragraph{Homework (20\%)}\label{homework-20}}

There are ten homeworks, assigned on Tuesdays and due 9 days later on
the following Thursday. The homeworks focus on interpreting results,
complete data analyses, and reinforce concepts and methods from lecture
and lab. Free free to discuss homework assignments with other students
-- however, all work must be your own and submitted individually.

\emph{No late homeworks are accepted, but the lowest grade is
automatically dropped.}

\hypertarget{labs-10}{%
\paragraph{Labs (10\%)}\label{labs-10}}

There are ten computing labs, assigned on Mondays and due 3 days later
that Thursday (except the first lab). Labs are primarily team-based work
that focus on developing programming tools to tackle analysis of
real-world datasets. Lab groups will be assigned in order to promote
diversity among team members -- you will be asked to evaluate each
others' contributions periodically throughout the semester.

\emph{No late labs are accepted, but the lowest grade is automatically
dropped.}

\hypertarget{midterm-exams-45}{%
\paragraph{Midterm Exams (45\%)}\label{midterm-exams-45}}

Three midterm exams test understanding and interpretation of methods,
with inclusion of some basic computations as needed. Each exam
corresponds to one of the three units, and are scheduled on Tuesdays
(with the lecture on the previous Thursday being reserved for in-class
review of tested concepts).

All exams are closed, in-class exams, with \textbf{absolutely no
electronics allowed}, including calculators. However, one (1) single 3''
by 5'' handwritten index card (front and back) is permitted for use as
notes to be used for the exam; these must be turned in with the exam.

\emph{Exam dates cannot be changed and no make-up exams will be given.}

\hypertarget{final-group-project-20}{%
\paragraph{Final Group Project (20\%)}\label{final-group-project-20}}

The final project is an open-ended statistical analysis that answers a
research question of interest using a real-world dataset. The project
must be completed with lab group members; you will be asked to evaluate
team members' contributions. A proposal is assigned March 31 (due April
7), and final write-ups are due Sunday, April 26 at 11:59p, with
in-class presentations from 9--12 on Monday, April 27. Submission must
be made via electronic submission to the class Sakai page, with
unlimited re-submissions allowed until 11:59p on April 26.

\emph{You must turn in a write-up on time and present with your lab
group during the final exam period on April 27 in order to pass the
course.}

More details to follow.

\hypertarget{participation-5}{%
\paragraph{Participation (5\%)}\label{participation-5}}

Participation consists of completion of in-class activities, as tracked
by answering discussion questions via Google Forms, in order for me to
gauge class understanding and stimulate discussion. Participation is
checked for completeness, not accuracy. Completing 90\% or more of
participation activities will count as full participation; completing
less than 90\% will result in the participation grade being assessed pro
rata according to the percentage of activities completed

\hypertarget{grade-calculation}{%
\subsubsection{Grade Calculation}\label{grade-calculation}}

The following table presents the contribution of each component to a
student's final grade:

\{.table .table2 .table-condensed .table-striped .text-left\} {}
\textbar{} {}\\
-----\textbar---- Homework \textbar{} 20\% Labs \textbar{} 10\% Midterm
Exam 1 \textbar{} 15\% Midterm Exam 2 \textbar{} 15\% Midterm Exam 3
\textbar{} 15\% Final Project \textbar{} 20\% Participation \textbar{}
5\%

A letter grade will be assigned as follows:

%\{.table .table2 .table-condensed .table-striped .text-center\} {}
%\textbar{} {} \textbar{} {} \textbar{} {} \textbar{} {}\\
%----\textbar----\textbar----\textbar----\textbar---- 92.5 \textbar{} ≤
%\textbar{} A \textbar{} ≤ \textbar{} 100.0 90.0 \textbar{} ≤ \textbar{}
%A- \textbar{} \textless{} \textbar{} 92.5 87.5 \textbar{} ≤ \textbar{}
%B+ \textbar{} \textless{} \textbar{} 90.0 82.5 \textbar{} ≤ \textbar{} B
%\textbar{} \textless{} \textbar{} 87.5 80.0 \textbar{} ≤ \textbar{} B-
%\textbar{} \textless{} \textbar{} 82.5 77.5 \textbar{} ≤ \textbar{} C+
%\textbar{} \textless{} \textbar{} 80.0 72.5 \textbar{} ≤ \textbar{} C
%\textbar{} \textless{} \textbar{} 77.5 70.0 \textbar{} ≤ \textbar{} C-
%\textbar{} \textless{} \textbar{} 72.5 67.5 \textbar{} ≤ \textbar{} D+
%\textbar{} \textless{} \textbar{} 70.0 62.5 \textbar{} ≤ \textbar{} D
%\textbar{} \textless{} \textbar{} 67.5 60.0 \textbar{} ≤ \textbar{} D-
%\textbar{} \textless{} \textbar{} 62.5 0.0 \textbar{} ≤ \textbar{} F
%\textbar{} \textless{} \textbar{} 60.0

Attendance in lecture and lab is a firm expectation; frequent absences
or tardiness will be considered a legitimate cause for grade reduction.

If you have a cumulative numerical average of at least 90\%, you are
guaranteed at least an A-; 80\% at least a B-, etc. I may use more
generous cut points depending on the overall difficulty of assignments
and examinations, but under no circumstances will a student's grade be
curved downward.

Regrade requests for homeworks, labs, and exams must be submitted within
one week of when the assignment is returned, and should be submitted
through the \href{link\%20goes\%20here}{online form}. Regrade requests
will be honored if there is an error in the grade calculation or a
correct answer was mistakenly marked as incorrect. Note, however, that
regrades may result in a lower grade than originally given. No grades
will be changed after the final project presentations.

\emph{Grades can only be changed by Professor Jiang. Teaching Assistants
cannot change grades on returned assignments.}

\hypertarget{absences-and-late-work}{%
\subsubsection{Absences and Late Work}\label{absences-and-late-work}}

Students who miss a class due to a scheduled varsity trip, religious
holiday, or short-term illness should fill out the respective form:
however, these excused absences do not excuse you from assigned work.

If you have a personal or family emergency or chronic health condition
that affects your ability to participate in class, please contact your
academic dean's office. Review the
\href{https://trinity.duke.edu/undergraduate/academic-policies/class-attendance-and-missed-work}{Trinity
excused absence policy} for further details.

Exams missed in accordance with the excused absence policy will not
count toward grade calculations; the other two exams will be
up-weighted. However, you must take at least two exams, submit the final
project write-up on time, and present the final project during the final
exam period in order to pass the course.

\textbf{No late work is accepted.}

\hypertarget{procedure-for-testing-accommodations}{%
\subsubsection{Procedure for Testing
Accommodations}\label{procedure-for-testing-accommodations}}

This class will use the Testing Center to provide testing accommodations
to undergraduates registered with and approved by the
\href{https://access.duke.edu/students}{Student Diability Access Office
(SDAO)}. The center operates by appointment only and appointments must
be made at least 7 consecutive days in advance, but please schedule your
appointments as far in advance as possible. You will not be able to make
an appointment until you have submitted a Semester Request with the SDAO
and it has been approved. So, if you have not done so already, promptly
submit a Semester Request to the SDAO in order to make your appointment
in time. For instructions on how to register with SDAO, visit their
website \href{https://access.duke.edu/requests}{here}. For instructions
on how to make an appointment at the Testing Center, visit their website
\href{https://testingcenter.duke.edu}{here}.

\hypertarget{inclusion-class-etiquette}{%
\subsubsection{Inclusion \& Class
Etiquette}\label{inclusion-class-etiquette}}

Behavior in and out of the classroom should enhance the learning
process. At all times we will use common courtesy and respectful
behavior. In this course, we will strive to create a learning
environment that is welcoming to all students and that is in alignment
with
\href{https://provost.duke.edu/initiatives/commitment-to-diversity-and-inclusion}{Duke's
Commitment to Diversity and Inclusion}.

If there is any aspect of the class that is not welcoming or accessible
to you, please let me know immediately. Additionally, if you are
experiencing something outside of class that is affecting your
performance in the course, please feel free to talk with me and/or your
academic dean

\hypertarget{where-to-find-help}{%
\subsubsection{Where to find help}\label{where-to-find-help}}

If you have a question, ask! There are likely other students with the
same question, so by asking you will create a learning opportunity for
everyone. Occasionally, I may defer a question to office hours. Please
be understanding -- it does not mean that I think your question is bad;
we may simply be running behind. - Office hours are a valuable resource
for more individual attention. Use them! - Outside of class and office
hours, general questions about course content or assignments should be
posted on the course Piazza site, since there are likely other students
with the same questions. - Sometimes you may need help with the class
that is beyond what can be provided by the teaching team. In that
instance, I encourage you to visit the
\href{https://arc.duke.edu/}{Academic Resource Center (ARC)}. The ARC
offers free services to all students during their undergraduate careers
at Duke. Services include Learning Consultations, Peer Tutoring and
Study Groups, ADHD/LD Coaching, Outreach Workshops, and more.

\hypertarget{academic-honesty}{%
\subsubsection{Academic Honesty}\label{academic-honesty}}

Academic honesty is of paramount importance in this class, and all work
must be done in accordance with the Duke Community Standar, reproduced
as follows:

To uphold the
\href{https://studentaffairs.duke.edu/conduct/about-us/duke-community-standard}{Duke
Community Standard}: - I will not lie, cheat, or steal in my academic
endeavors; - I will conduct myself honorably in all my endeavors; and -
I will act if the Standard is compromised.

By enrolling in this course, you have agreed to abide by and uphold the
provisions of the
\href{https://studentaffairs.duke.edu/conduct/about-us/duke-community-standard}{Duke
Community Standard} as well as the policies specific to this course. Any
violations will automatically result in a grade of 0 on the assignment
and will be reported to the
\href{https://studentaffairs.duke.edu/conduct}{Office of Student
Conduct} for further action.

\hypertarget{technology}{%
\subsubsection{Technology}\label{technology}}

You should bring a laptop to every lecture and lab session. Outlets are
limited, so make sure it is fully-charged. Ensure the volume on all
devices is set to mute, and please refrain from engaging in activities
not related to the class discussion. Browsing the web and social media,
excessive messaging, playing games, etc. is not only a distraction for
you but is also a distraction for everyone around you.

\end{document}
